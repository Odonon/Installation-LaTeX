%% Cet exemple trace des courbes de référence de Holman pour l'efficacité d'une ailette cylindrique.
%% Pour compiler, il nécessite le document texte ailette_fusee_holman.txt
%% Le but de cet exemple est de montrer l'utilisation simple de l'environnement axis et des nodes dans les tikzpicture
\documentclass{standalone}
\usepackage{amsmath}
\usepackage{pgfplots}
	\pgfplotsset{compat=newest}
	\usetikzlibrary{quotes,angles,patterns,plotmarks,external}
\usepackage{xfrac}
	\newcommand*\rfrac[2]{{}^{#1}\!/_{#2}}%running fraction with slash - requires math mode. Chris H (TeXchange)
\begin{document}
\begin{tikzpicture}
\tikzset{%
	every pin/.style={fill=white,rectangle},
	small dot/.style={fill=black,circle,scale=0.3}
}
\begin{axis}[%
	xlabel={${L_c}^{\rfrac{3}{2}}\sqrt{\dfrac{h_{conv}}{k\,A_m}}$},
	ylabel={Efficacité de l'ailette $\eta_{ail}$},
	ymax=1,
	ymin=0,
	xmax=3,
	xmin=0,
	xmajorgrids,
	ymajorgrids
	]
\addplot+[no marks,line width=1pt,black] table[x=x,y=Curve1] {ailette_fusee_holman.txt};
\addplot+[no marks,line width=1pt,black] table[x=x,y=Curve2] {ailette_fusee_holman.txt};
\addplot+[no marks,line width=1pt,black] table[x=x,y=Curve3] {ailette_fusee_holman.txt};
\addplot+[no marks,line width=1pt,black] table[x=x,y=Curve4] {ailette_fusee_holman.txt};
\addplot+[no marks,line width=1pt,black] table[x=x,y=Curve5] {ailette_fusee_holman.txt};
\node[pin=45:{$\dfrac{r_{2c}}{r_1}=1$}] at (axis description cs:0.22,0.75) {};
\node[pin=45:{$2$}] at (axis description cs:0.3,0.55) {};
\node[pin=45:{$3$}] at (axis description cs:0.35,0.43) {};
\node[pin=225:{$4$}] at (axis description cs:0.5,0.32) {};
\node[pin=260:{$5$}] at (axis description cs:0.6,0.23) {};
\end{axis}
\end{tikzpicture}
\end{document}