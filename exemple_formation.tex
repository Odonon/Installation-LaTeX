\documentclass[11pt,a4paper]{article}
\usepackage[utf8]{inputenc}
\usepackage[T1]{fontenc}
\usepackage[french]{babel} % permet la mise en page à la française
\usepackage{amsmath} % insertion de certains symboles mathématiques
\usepackage{amsfonts} % insertion de certains symboles mathématiques
\usepackage{amssymb} % insertion de certains symboles mathématiques
\usepackage{graphicx} % insertion d'images
\usepackage{pgfplots} % pour faire des beaux graphiques
\usepackage{standalone}

\begin{document}
	La taille des polices se fait par les commandes suivantes:
	\begin{itemize}
		\item \begin{verbatim}
		\tiny
		\end{verbatim}
		{\tiny C'est tout petit}
		
		\item \begin{verbatim}
		\small
		\end{verbatim}
		{\small C'est petit}
		
		\item \begin{verbatim}
		\normalsize
		\end{verbatim}
		{\normalsize C'est normal}
		
		\item \begin{verbatim}
		\large
		\end{verbatim}
		{\large C'est plus grand}
		
		\item \begin{verbatim}
		\Large
		\end{verbatim}
		{\Large C'est très grand}
		
		\item \begin{verbatim}
		\LARGE
		\end{verbatim}
		{\LARGE C'est très très grand}
		
		\item \begin{verbatim}
		\huge
		\end{verbatim}
		{\huge C'est énorme}
		
		\item \begin{verbatim}
		\Huge
		\end{verbatim}
		{\Huge C'est vraiment énorme}
		\newline
	\end{itemize}


	Les styles comme italiques, gras etc.:
	\begin{itemize}
		\item \begin{verbatim}
		\textit{contenu...}
		\end{verbatim}
		\textit{Du texte en italique}
		
		\item \begin{verbatim}
		\textbf{contenu...}
		\end{verbatim}
		\textbf{Du texte en gras}
		
		\item \begin{verbatim}
		\underline{contenu...}
		\end{verbatim}
		\underline{Du texte souligné}
		\newline
	\end{itemize}

Je veux insérer une image et la centrer dans une figure \ref{fig:big_photo}.

\begin{verbatim}
\begin{figure}
	\centering
	\includegraphics{photo.jpg}
	\caption{une belle photo trop grande}
	\label{fig:big_photo}
\end{figure}
\end{verbatim}
\begin{figure}
	\centering
	\includegraphics{photo.jpg}
	\caption{une belle photo trop grande}
	\label{fig:big_photo}
\end{figure}

Pour la redimensionner, on peut utiliser les options de \textit{includegraphics}. Comme sur la figure \ref{fig:not_big_photo}.
\begin{verbatim}
\begin{figure}
	\centering
	\includegraphics[width=\linewidth]{photo.jpg}
	\caption{une belle photo moins grande}
	\label{fig:not_big_photo}
\end{figure}
\end{verbatim}
\begin{figure}
	\centering
	\includegraphics[width=\linewidth]{photo.jpg}
	\caption{une belle photo moins grande}
	\label{fig:not_big_photo}
\end{figure}
\end{document}